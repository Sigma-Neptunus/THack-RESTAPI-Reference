\documentclass{xoblivoir}
\usepackage{kotex}

\title{Neptunus-THackAThon\\
\large{RESTful API Reference}}
\author{Junyoung Clare Jang\\
\normalsize{v0.0.1}}
\begin{document}
\maketitle
\section{산 정보}
\begin{enumerate}
\item \emph{SERVER}/mountains[/?order=string\&start=number\&length=number]
  
  order: nameasc \textbar{} namedec \textbar{} heightasc \textbar{} heightdec \textbar{} distanceasc \textbar{} distancedec\\
  start: 0 \textbar{} any positive numbers\\
  length: 0 \textbar{} any positive numbers
  \begin{enumerate}
  \item GET (Authority: For all)
    
    산들에 대한 정보를 돌려준다. Query 가 주어졌다면 order 에 명시된 방식으로 정렬한 뒤, start 번째의 산부터 length 개의 산들에 대한 정보를 Entity Body 로 돌려준다.
    만일 start 가 서버에 저장된 산들의 개수보다 크다면, 404 Not found 를 돌려준다.
    만일 start 가 올바르고, length 가 0이거나 서버에 저장된 산들의 개수보다 크다면, start 번째의 산부터 모든 산들에 대한 정보를 Entity Body 로 돌려준다.
  \end{enumerate}

\item \emph{SERVER}/mountains/\textbf{\#id}

  \#id= any valid id
  \begin{enumerate}
    \item GET (Authority: For all)
      id에 해당하는 산에 대한 정보를 
    \item POST (Authority: For admin)

    \item PUT (Authority: For admin)

    \item DELETE (Authority: For admin)
  \end{enumerate}
\end{enumerate}
\end{document}